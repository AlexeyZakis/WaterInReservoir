\documentclass[oneside,senior,etd]{BYUPhys}

\usepackage{cmap} % Для корректной кодировки в pdf
\usepackage[utf8]{inputenc}
\usepackage{rotating}

\usepackage[russian]{babel}
\usepackage{amsfonts} % Пакеты для математических символов и теорем
\usepackage{amstext}
\usepackage{amssymb}
\usepackage{amsthm}
\usepackage{graphicx} % Пакеты для вставки графики
\usepackage{subfig}
\usepackage{color}
\usepackage[unicode]{hyperref}
\usepackage[nottoc]{tocbibind} % Для того, чтобы список литературы отображался в оглавлении
\usepackage{verbatim} % Для вставок заранее подготовленного текста в режиме as-is
\usepackage{listings}

\usepackage{tikz}
\usepackage{pgfplots}
\usetikzlibrary{arrows,positioning}
\usepackage{adjustbox}

\usepackage{makecell}
\usepackage{booktabs}
\usepackage{boldline}

\usepackage{xcolor}
\usepackage{soul}
\usepackage{url}
\usepackage{multirow}
\usepackage{amsmath}

\usepackage{pifont}
\usepackage{indentfirst} % Делать отступ в начале первого параграфа

\usepackage{caption}

\usepackage{algpseudocode}

\usepackage{placeins}

\usepackage{perpage}
\MakePerPage{footnote} % Сброс счётчиков сносок для каждой страницы

% Общие параметры листингов
\lstset{
	%frame=TB,
	showstringspaces=false,
	tabsize=4,
	basicstyle=\linespread{1.0}\tt\small, % делаем листинги компактнее
	breaklines=false,
	texcl=true, % русские буквы в комментариях
	captionpos=b,
	aboveskip=\baselineskip,
	commentstyle=\tt
}

% Стиль кода

\lstloadlanguages{C, C++, csh}

\definecolor{backgroundColor}{rgb}{0.93, 0.93, 0.93}
\definecolor{commentColor}{rgb}{0.4, 0.4, 0.6}
\definecolor{keyWordsColor}{rgb}{0.4, 0.4, 1}
\definecolor{identifiersColor}{rgb}{0.4, 0.2, 0.6}

\lstset{
	language=csh,
	basicstyle=\footnotesize\ttfamily,
	numbers=left,
	numberstyle=\tiny,
	numbersep=5pt,
	tabsize=2,
	extendedchars=true,
	breaklines=true,
	frame=b,
	stringstyle=\color{blue}\ttfamily,
	showspaces=false,
	showtabs=false,
	xleftmargin=17pt,
	framexleftmargin=17pt,
	framexrightmargin=5pt,
	framexbottommargin=4pt,
	commentstyle=\color{commentColor},
	morecomment=[l]{//}, %use comment-line-style!
	morecomment=[s]{/*}{*/}, %for multiline comments
	showstringspaces=false,
	morekeywords={ abstract, event, new, struct,
		as, explicit, null, switch,
		base, extern, object, this,
		bool, false, operator, throw,
		break, finally, out, true,
		byte, fixed, override, try,
		case, float, params, typeof,
		catch, for, private, uint,
		char, foreach, protected, ulong,
		checked, goto, public, unchecked,
		class, if, readonly, unsafe,
		const, implicit, ref, ushort,
		continue, in, return, using,
		decimal, int, sbyte, virtual,
		default, interface, sealed, volatile,
		delegate, internal, short, void,
		do, is, sizeof, while,
		double, lock, stackalloc,
		else, long, static,
		enum, namespace, string},
	keywordstyle=\color{keyWordsColor},
	identifierstyle=\color{identifiersColor},
	backgroundcolor=\color{backgroundColor},
	escapeinside={(*}{*)} % Разрешаем вставку команд LaTeX между "(*" и "*)"
}


%\addto\captionsrussian{ % т.к. babel переопределяет \lstlistingname
	%	\renewcommand{\lstlistingname}{Пример кода} % Подписи листингов
	%	\renewcommand{\lstlistlistingname}{Список примеров кода} % Подписи списка листингов
	%}

\newcommand{\todo}[1]{\textcolor{red}{#1}}

\usepackage[normalem]{ulem} % Чтобы зачёркивать текст

\definecolor{editColor}{RGB}{164, 46, 255}
\newcommand{\lastAdded}[1]{\textcolor{editColor}{\textbf{#1}}} % Чтобы отмечать добавленный текст в последней версии
\newcommand{\lastRemoved}[1]{\textcolor{editColor}{\sout{#1}}} % Чтобы отмечать удалнный текст в последней версии

% Цвет текста для акцентирования внимания
\definecolor{importantTextColor}{rgb}{0.8, 0.2, 0.2}

\usepackage{tcolorbox}

\newcommand{\highlight}[1]{%
	\textcolor{importantTextColor}{\fontsize{12}{11}\ttfamily\linespread{1.0}#1}%
}

\usepackage[table]{colortbl} % Для использования цвета в таблицах

% Определение новой команды для форматирования определений терминов
\newcommand{\definition}[2]{\textbf{#1} --- #2\par}

% DEBUG
% \usepackage{showframe}

\Faculty{Факультет вычислительной математики и кибернетики}
\Chair{Кафедра алгоритмических языков}
% Лаборатория
% \Lab{~}
\Year{2025}
\Month{Ноябрь}
\City{Москва}
\AuthorText{Автор:}
\Author{Плотников Алексей Сергеевич}
\AuthorEng{Alexey Plotnikov}
\AcadGroup{134}

\TitleTop{Математическая модель изменения}
% Раскомментируйте, если нужны еще строчки названия
\TitleMiddle{уровня воды в водохранилище}
%\TitleBottom{с помощью статического анализа}
% Uncomment if you need English title
% \TitleTopEng{Thesis theme, first line}
% \TitleMiddleEng{Thesis theme, second line}
% \TitleBottomEng{Thesis theme, third line}

\docname{Реферат}
\Advisor{Сычугов Дмитрий Юрьевич}
\AdvisorDegree{}
% Раскомментируйте, чтобы добавить научного консультанта
%\Consultant{Игнатьев Валерий Николаевич}
%\ConsultantDegree{к.ф.-м.н.\\}

% Закомментируйте, если аннотация не нужна
%\Abstract{}
% Раскомментируйте, если нужна английская аннотация
% \AbstractEng{Abstract in English}

% Раскомментируйте, чтобы написать благодарности
% \Acknowledgments{Благодарности.}

%%%% DON'T change this. It is here because .sty does not support cyrillic cp properly %%%%
\TitlePageText{Титульная страница}
\University{Московский государственный университет имени М.В.Ломоносова}
\GrText{гр.}
\AdvisorText{Преподаватель}
\ConsultantText{Научный консультант}
\AbstractText{Аннотация}
\AcknowledgmentsText{Благодарности}
\ListingText{Листинг}
\AlgorithmText{Алгоритм}

% Set PDF title and author
\hypersetup{
	pdftitle={\PDFTitle},
	pdfauthor={\PDFAuthor}
}

\begin{document}
	
	\fixmargins
	\makepreliminarypages
	
	\oneandhalfspace
	
	\pdfbookmark[section]{\contentsname}{toc}
	\tableofcontents
	
	\clearpage
	
	\section{Введение}
	
	Целью данной работы является построение и анализ математической модели, описывающей изменение уровня воды в водохранилище. Водохранилище представляет собой искусственный водоём, объём воды в котором зависит от множества физических процессов:
	
	\begin{itemize}
		\item приток воды из питающих рек и подземных источников;
		\item отток воды через плотину, водослив или гидротурбины;
		\item атмосферные осадки на поверхность водоёма;
		\item испарение воды с открытой поверхности;
		\item фильтрационные потери через почву.
	\end{itemize}
	
	\section{Физическая постановка задачи}
	
	Пусть $V(t)$~--- объём воды в водохранилище в момент времени $t$. Тогда общий баланс масс можно записать в виде:
	\begin{equation}
		\frac{dV}{dt} = Q_{\text{in}}(t) - Q_{\text{out}}(t) 
		+ Q_{\text{atm}}(t) - Q_{\text{loss}}(t).
		\label{eq:mass_balance}
	\end{equation}
	
	Здесь:
	\begin{itemize}
		\item $Q_{\text{in}}(t)$~--- приток воды в водохранилище;
		\item $Q_{\text{out}}(t)$~--- отток через плотину;
		\item $Q_{\text{atm}}(t)$~--- влияние атмосферы (осадки минус испарение);
		\item $Q_{\text{loss}}(t)$~--- фильтрационные потери (количество воды уходящее через почву).
	\end{itemize}
	
	Фильтрационные потери зависят от уровня воды $h(t)$:
	\begin{equation}
		Q_{\text{loss}}(t) = k_f h(t),
		\label{eq:filtration}
	\end{equation}
	где $k_f$~--- коэффициент фильтрации.
	
	Отток воды \(Q_{\text{out}}(t)\), в основном, происходит через плотину. В общем случае расход может включать
	как контролируемый сброс \(Q_{\text{ctrl}}(t)\) через затворы или турбины, так и естественный перелив \(Q_{\text{spill}}(t)\):
	
	\begin{equation}
		Q_{\text{out}}(t) = Q_{\text{ctrl}}(t) + Q_{\text{spill}}(t).
	\end{equation}
	
	\begin{itemize}
		\item $Q_{\text{ctrl}}(t)$ \text{ --- управляемый расход воды заданный оператором станции};
		\item $Q_{\text{spill}}(t)$ \text{ ---  поток воды сбрасываемый при привышении уровня воды}.
	\end{itemize}
	
	Влияние атмосферы определяется площадью зеркала водоёма $A(t)$:
	\begin{equation}
		Q_{\text{atm}}(t) = A(t)\bigl(P(t) - E(t)\bigr),
		\label{eq:atm}
	\end{equation}
	где $P(t)$~--- интенсивность осадков, \\
	$E(t)$~--- интенсивность испарения.
	
	Пусть зависимость объёма водохранилища от уровня воды аппроксимируется степенной функцией:
	\begin{equation}
		V(h)=\alpha h^n.
		\label{eq:V_h}
	\end{equation}
	
	Тогда площадь зеркала определяется:
	\begin{equation}
		A(h)=\frac{dV}{dh}=\alpha n h^{n-1}.
		\label{eq:A_h}
	\end{equation}
	
	Подставляя \eqref{eq:A_h} в выражение \eqref{eq:atm}, получим:
	\begin{equation}
		Q_{\text{atm}}(t)=\alpha n h^{n-1}\bigl(P(t)-E(t)\bigr).
		\label{eq:atm2}
	\end{equation}
	
	\section{Вывод основного уравнения модели}
	
	Подставим выражения \eqref{eq:atm2}, \eqref{eq:filtration} в баланс \eqref{eq:mass_balance} и учтём \eqref{eq:V_h}. Производная объёма:
	
	\begin{equation}
		\frac{dV}{dt}=\frac{d(\alpha h^n)}{dt}
		=\alpha n h^{n-1}\frac{dh}{dt}.
	\end{equation}
	
	Тогда основное уравнение принимает вид:
	\begin{equation}
		\alpha n h^{n-1}\frac{dh}{dt} =
		Q_{\text{in}}(t) - Q_{\text{ctrl}}(t)
		- Q_{\text{spill}}(t)
		+\alpha n h^{n-1}[P(t)-E(t)]
		-k_f h.
	\end{equation}
	
	После деления на $\alpha n h^{n-1}$ получим уравнение динамики уровня воды:
	\begin{equation}
		\frac{dh}{dt}=
		\frac{Q_{\text{in}}(t) - Q_{\text{ctrl}}(t) - Q_{\text{spill}}(t) - k_f h}{\alpha n h^{n-1}}
		+[P(t)-E(t)].
		\label{eq:main}
	\end{equation}
	
	\section{Сравнение модели с реальными данными}
	
	Для оценки корректности модели рассмотрим иркутское водохранилище
	
	Площадь водохранилища: $ A_0 = 154~\text{км}^2$~\cite{IrkutskReservoir}
	
	Другие параметры:
	\begin{itemize}
		\item Ежемесячное количество осадков~\cite{Precipitation}
		\item Ежемесячная величина испарения~\cite{Evaporation}
		\item Естественном приток воды~\cite{IrkutskReservoirWaterLevel}
		\item Средний расход воды~\cite{IrkutskReservoirWaterLevel}
	\end{itemize}
	
	\begin{figure}[h!]
		\centering
		\includegraphics[width=\linewidth]{results.png}
		\caption{Сравнение модели с реальными данными}
		\label{fig:results}
	\end{figure}
	
	\begin{itemize}
		\item Максимальное отклонение: 10.41см
		\item Стандартное отклонение: 3.57см
		\item Среднеквадратичная ошибка: 3.59см
	\end{itemize}
	
	Отклонение от реальных данным может быть вызвано:
	\begin{itemize}
		\item Использование усреднённых значений осадков и испарения
		\item Аппроксимация объёма воды степенной функцией от высоты
	\end{itemize}
	
	\section{Заключение}
	
	В работе построена математическая модель изменения уровня воды в водохранилище. Учтены основные физические процессы: приток, отток, осадки, испарение и фильтрация. Получено дифференциальное уравнение, описывающее динамику уровня воды.
	
	Полученная модель была сравнена с реальными данными Иркутского водохранилища. Среднеквадратичная ошибка модели от реальных данных составила 3.59см.
	
	\bibliographystyle{gost780u.bst} % Для соответствия требованиям об оформлении списка литературы
	\raggedright
	\bibliography{references}
	
	% Раскомментируйте, если нужно приложение
	% \appendix
	
	% \cleardoublepage \phantomsection
	% \section*{Приложение}
	% \addcontentsline{toc}{section}{Приложение}
	
\end{document}
